% -------------------------
% Custom Title Page
% -------------------------
\begin{titlepage}
\centering

\vspace*{2cm}

{\Huge\bfseries Big Beautiful Book \par}
\vspace{0.4cm}
{\Large A book \par}

\vspace{1.2cm}
\includegraphics[width=0.33\textwidth]{lib/img/Revolving_circles.480x480_white.png}

\vspace{1.2cm}
{\Large John Haverlack \par}

\vfill

\end{titlepage}

% ----- COPYRIGHT / META PAGE -----
\newpage
\thispagestyle{empty}
\begin{center}

\vspace*{\fill}

\textbf{Author}: John Haverlack

\textbf{Copyright} © 2025 John Haverlack

\textbf{License}: CC BY-ND 4.0

\textbf{Version}: 0.0.1

\textbf{Date}: 2025-11-05

\vspace*{\fill}

\end{center}

% ----- Title Page Image Note -----
\newpage
\thispagestyle{empty}

\vspace*{1.5cm}

\begin{center}
\includegraphics[width=0.33\textwidth]{lib/img/Revolving_circles.480x480_white.png}

\vspace{0.4cm}

\textit{“Revolving Circles.” n.d. Accessed October 31, 2025.} \\
\url{https://en.wikipedia.org/wiki/File/Revolving_circles.svg}
\end{center}

\vspace{1cm}

\noindent
The cover image presents a visual illusion of motion. When you focus on the central
point and move the page toward or away from your eyes, the concentric circles appear
to rotate. This effect reflects the theme of perception and the mind's role in
constructing reality.

\medskip

\noindent
This image serves as a simple demonstration that what our brains perceive can vary
significantly from underlying reality. In this case, we perceive motion where none
exists. The point is that the best we can do in the human condition is to perceive,
not directly know, reality.

\vspace*{\fill}

\clearpage
\pagenumbering{arabic}
\setcounter{page}{1}