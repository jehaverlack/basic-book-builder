% Options for packages loaded elsewhere
\PassOptionsToPackage{unicode}{hyperref}
\PassOptionsToPackage{hyphens}{url}
\PassOptionsToPackage{dvipsnames,svgnames,x11names}{xcolor}
%
\documentclass[
  12pt,
]{article}
\usepackage{amsmath,amssymb}
\usepackage{lmodern}
\usepackage{iftex}
\ifPDFTeX
  \usepackage[T1]{fontenc}
  \usepackage[utf8]{inputenc}
  \usepackage{textcomp} % provide euro and other symbols
\else % if luatex or xetex
  \usepackage{unicode-math}
  \defaultfontfeatures{Scale=MatchLowercase}
  \defaultfontfeatures[\rmfamily]{Ligatures=TeX,Scale=1}
  \setmainfont[]{Linux Libertine O}
  \setsansfont[]{Linux Biolinum O}
  \setmonofont[]{DejaVu Sans Mono}
\fi
% Use upquote if available, for straight quotes in verbatim environments
\IfFileExists{upquote.sty}{\usepackage{upquote}}{}
\IfFileExists{microtype.sty}{% use microtype if available
  \usepackage[]{microtype}
  \UseMicrotypeSet[protrusion]{basicmath} % disable protrusion for tt fonts
}{}
\makeatletter
\@ifundefined{KOMAClassName}{% if non-KOMA class
  \IfFileExists{parskip.sty}{%
    \usepackage{parskip}
  }{% else
    \setlength{\parindent}{0pt}
    \setlength{\parskip}{6pt plus 2pt minus 1pt}}
}{% if KOMA class
  \KOMAoptions{parskip=half}}
\makeatother
\usepackage{xcolor}
\IfFileExists{xurl.sty}{\usepackage{xurl}}{} % add URL line breaks if available
\IfFileExists{bookmark.sty}{\usepackage{bookmark}}{\usepackage{hyperref}}
\hypersetup{
  colorlinks=true,
  linkcolor={blue},
  filecolor={Maroon},
  citecolor={Blue},
  urlcolor={Blue},
  pdfcreator={LaTeX via pandoc}}
\urlstyle{same} % disable monospaced font for URLs
\usepackage[margin=1in]{geometry}
\usepackage{color}
\usepackage{fancyvrb}
\newcommand{\VerbBar}{|}
\newcommand{\VERB}{\Verb[commandchars=\\\{\}]}
\DefineVerbatimEnvironment{Highlighting}{Verbatim}{commandchars=\\\{\}}
% Add ',fontsize=\small' for more characters per line
\newenvironment{Shaded}{}{}
\newcommand{\AlertTok}[1]{\textcolor[rgb]{1.00,0.00,0.00}{\textbf{#1}}}
\newcommand{\AnnotationTok}[1]{\textcolor[rgb]{0.38,0.63,0.69}{\textbf{\textit{#1}}}}
\newcommand{\AttributeTok}[1]{\textcolor[rgb]{0.49,0.56,0.16}{#1}}
\newcommand{\BaseNTok}[1]{\textcolor[rgb]{0.25,0.63,0.44}{#1}}
\newcommand{\BuiltInTok}[1]{\textcolor[rgb]{0.00,0.50,0.00}{#1}}
\newcommand{\CharTok}[1]{\textcolor[rgb]{0.25,0.44,0.63}{#1}}
\newcommand{\CommentTok}[1]{\textcolor[rgb]{0.38,0.63,0.69}{\textit{#1}}}
\newcommand{\CommentVarTok}[1]{\textcolor[rgb]{0.38,0.63,0.69}{\textbf{\textit{#1}}}}
\newcommand{\ConstantTok}[1]{\textcolor[rgb]{0.53,0.00,0.00}{#1}}
\newcommand{\ControlFlowTok}[1]{\textcolor[rgb]{0.00,0.44,0.13}{\textbf{#1}}}
\newcommand{\DataTypeTok}[1]{\textcolor[rgb]{0.56,0.13,0.00}{#1}}
\newcommand{\DecValTok}[1]{\textcolor[rgb]{0.25,0.63,0.44}{#1}}
\newcommand{\DocumentationTok}[1]{\textcolor[rgb]{0.73,0.13,0.13}{\textit{#1}}}
\newcommand{\ErrorTok}[1]{\textcolor[rgb]{1.00,0.00,0.00}{\textbf{#1}}}
\newcommand{\ExtensionTok}[1]{#1}
\newcommand{\FloatTok}[1]{\textcolor[rgb]{0.25,0.63,0.44}{#1}}
\newcommand{\FunctionTok}[1]{\textcolor[rgb]{0.02,0.16,0.49}{#1}}
\newcommand{\ImportTok}[1]{\textcolor[rgb]{0.00,0.50,0.00}{\textbf{#1}}}
\newcommand{\InformationTok}[1]{\textcolor[rgb]{0.38,0.63,0.69}{\textbf{\textit{#1}}}}
\newcommand{\KeywordTok}[1]{\textcolor[rgb]{0.00,0.44,0.13}{\textbf{#1}}}
\newcommand{\NormalTok}[1]{#1}
\newcommand{\OperatorTok}[1]{\textcolor[rgb]{0.40,0.40,0.40}{#1}}
\newcommand{\OtherTok}[1]{\textcolor[rgb]{0.00,0.44,0.13}{#1}}
\newcommand{\PreprocessorTok}[1]{\textcolor[rgb]{0.74,0.48,0.00}{#1}}
\newcommand{\RegionMarkerTok}[1]{#1}
\newcommand{\SpecialCharTok}[1]{\textcolor[rgb]{0.25,0.44,0.63}{#1}}
\newcommand{\SpecialStringTok}[1]{\textcolor[rgb]{0.73,0.40,0.53}{#1}}
\newcommand{\StringTok}[1]{\textcolor[rgb]{0.25,0.44,0.63}{#1}}
\newcommand{\VariableTok}[1]{\textcolor[rgb]{0.10,0.09,0.49}{#1}}
\newcommand{\VerbatimStringTok}[1]{\textcolor[rgb]{0.25,0.44,0.63}{#1}}
\newcommand{\WarningTok}[1]{\textcolor[rgb]{0.38,0.63,0.69}{\textbf{\textit{#1}}}}
\usepackage{longtable,booktabs,array}
\usepackage{calc} % for calculating minipage widths
% Correct order of tables after \paragraph or \subparagraph
\usepackage{etoolbox}
\makeatletter
\patchcmd\longtable{\par}{\if@noskipsec\mbox{}\fi\par}{}{}
\makeatother
% Allow footnotes in longtable head/foot
\IfFileExists{footnotehyper.sty}{\usepackage{footnotehyper}}{\usepackage{footnote}}
\makesavenoteenv{longtable}
\usepackage{graphicx}
\makeatletter
\def\maxwidth{\ifdim\Gin@nat@width>\linewidth\linewidth\else\Gin@nat@width\fi}
\def\maxheight{\ifdim\Gin@nat@height>\textheight\textheight\else\Gin@nat@height\fi}
\makeatother
% Scale images if necessary, so that they will not overflow the page
% margins by default, and it is still possible to overwrite the defaults
% using explicit options in \includegraphics[width, height, ...]{}
\setkeys{Gin}{width=\maxwidth,height=\maxheight,keepaspectratio}
% Set default figure placement to htbp
\makeatletter
\def\fps@figure{htbp}
\makeatother
\setlength{\emergencystretch}{3em} % prevent overfull lines
\providecommand{\tightlist}{%
  \setlength{\itemsep}{0pt}\setlength{\parskip}{0pt}}
\setcounter{secnumdepth}{5}
\newlength{\cslhangindent}
\setlength{\cslhangindent}{1.5em}
\newlength{\csllabelwidth}
\setlength{\csllabelwidth}{3em}
\newlength{\cslentryspacingunit} % times entry-spacing
\setlength{\cslentryspacingunit}{\parskip}
\newenvironment{CSLReferences}[2] % #1 hanging-ident, #2 entry spacing
 {% don't indent paragraphs
  \setlength{\parindent}{0pt}
  % turn on hanging indent if param 1 is 1
  \ifodd #1
  \let\oldpar\par
  \def\par{\hangindent=\cslhangindent\oldpar}
  \fi
  % set entry spacing
  \setlength{\parskip}{#2\cslentryspacingunit}
 }%
 {}
\usepackage{calc}
\newcommand{\CSLBlock}[1]{#1\hfill\break}
\newcommand{\CSLLeftMargin}[1]{\parbox[t]{\csllabelwidth}{#1}}
\newcommand{\CSLRightInline}[1]{\parbox[t]{\linewidth - \csllabelwidth}{#1}\break}
\newcommand{\CSLIndent}[1]{\hspace{\cslhangindent}#1}
\usepackage{etoolbox}
\usepackage[most]{tcolorbox}
\usepackage{graphicx}
\usepackage{xparse}
\tcbuselibrary{breakable}

% --- Disable floating figures globally ---
\usepackage{float}
\let\origfigure\figure
\let\endorigfigure\endfigure
\renewenvironment{figure}[1][H]{\origfigure[H]}{\endorigfigure}

% --- Colors ---
\usepackage{xcolor}
\definecolor{speculativeback}{RGB}{247,232,255}  % #f7e8ff
\definecolor{speculativeframe}{RGB}{139,43,226}  % #8b2be2

% --- Left-align all Pandoc tables ---
\usepackage{longtable}
\usepackage{etoolbox}
\makeatletter
\patchcmd\longtable{\par}{\par\raggedright}{}{}
\setlength{\LTleft}{0pt}
\setlength{\LTright}{0pt}
\makeatother


% Helper to show a blank title if empty
\newcommand{\blanktitle}{\mbox{}}

% =========================================================
% Concept Boxes
% =========================================================
\newtcolorbox{infobox}[1][Information]{%
  title=\ifstrempty{#1}{\blanktitle}{#1},
  colback=blue!3!white,
  colframe=blue!50!black,
  colbacktitle=blue!10!white,
  fonttitle=\bfseries,
  coltitle=black,
  enhanced, sharp corners, breakable,
  after=\par\vspace{6pt}
}

\newtcolorbox{proposedbox}[1][Proposed Concept]{%
  title=\ifstrempty{#1}{\blanktitle}{#1},
  colback=blue!5!white,
  colframe=blue!75!black,
  colbacktitle=blue!15!white,
  fonttitle=\bfseries,
  coltitle=black,
  enhanced, sharp corners, breakable,
  after=\par\vspace{6pt}
}

\newtcolorbox{establishedbox}[1][Established Concept]{%
  title=\ifstrempty{#1}{\blanktitle}{#1},
  colback=green!5!white,
  colframe=green!75!black,
  colbacktitle=green!15!white,
  fonttitle=\bfseries,
  coltitle=black,
  enhanced, sharp corners, breakable,
  after=\par\vspace{6pt}
}

\newtcolorbox{speculativebox}[1][Speculative Concept]{%
  title=\ifstrempty{#1}{\blanktitle}{#1},
  colback=speculativeback,
  colframe=speculativeframe,
  colbacktitle=speculativeback!80!white, % lighter variant for title bar
  fonttitle=\bfseries,
  coltitle=black,
  enhanced, sharp corners, breakable,
  after=\par\vspace{6pt}
}


\newtcolorbox{cautionbox}[1][Caution]{%
  title=\ifstrempty{#1}{\blanktitle}{#1},
  colback=orange!5!white,
  colframe=orange!75!black,
  colbacktitle=orange!15!white,
  fonttitle=\bfseries,
  coltitle=black,
  enhanced, sharp corners, breakable,
  after=\par\vspace{6pt}
}

\newtcolorbox{warningbox}[1][Warning]{%
  title=\ifstrempty{#1}{\blanktitle}{#1},
  colback=orange!5!white,
  colframe=orange!75!black,
  colbacktitle=orange!15!white,
  fonttitle=\bfseries,
  coltitle=black,
  enhanced, sharp corners, breakable,
  after=\par\vspace{6pt}
}

\newtcolorbox{dangerbox}[1][Danger]{%
  title=\ifstrempty{#1}{\blanktitle}{#1},
  colback=red!5!white,
  colframe=red!75!black,
  colbacktitle=red!15!white,
  fonttitle=\bfseries,
  coltitle=black,
  enhanced, sharp corners, breakable,
  after=\par\vspace{6pt}
}
\ifLuaTeX
  \usepackage{selnolig}  % disable illegal ligatures
\fi

\author{}
\date{}

\begin{document}

\renewcommand*\contentsname{Contents}
{
\hypersetup{linkcolor=}
\setcounter{tocdepth}{3}
\tableofcontents
}
\hypertarget{introduction}{%
\section{Introduction}\label{introduction}}

This is a basic book (or article) builder template based on a Pandoc
build process in conjunction with a number of other tools to generate
PDF, ODT, HTML, LaTex, Markdown, and Epub book output formats from
Markdown source content , which can optionally be edited as an Obsidian
vault.

This book builder template has been curated by John
Haverlack.(\protect\hyperlink{ref-JohnHaverlackACEP}{{``John {Haverlack}
\textbar{} {ACEP}''} n.d.})

\hypertarget{conventions}{%
\subsection{Conventions}\label{conventions}}

A few callout box styles have been added to easily highlight content.

\begin{establishedbox}[Established Concept]
Einsteins Relativistic Dynamics Equations
\[E^2 = (m_{0} \cdot c^2)^2 + (p \cdot c)^2 \]
\end{establishedbox}

\begin{proposedbox}[Proposed Concept]
With the speed of light, \(c = 1\): \[E^2 = m_{0}^2 + p^2 \]
\end{proposedbox}

\begin{speculativebox}[Speculative Concept]
With the speed of light, \(c = 1\): \[E^2 = m_{0}^2 + p^2 \]
\end{speculativebox}

\begin{cautionbox}[Caution Note]
Beware of this section.
\end{cautionbox}

\begin{warningbox}[Warning Note]
Beware of this section.
\end{warningbox}

\begin{dangerbox}[Alerts]
Extreme Highlight
\end{dangerbox}

\hypertarget{getting-started}{%
\section{Getting Started}\label{getting-started}}

\hypertarget{installing-pre-requisites}{%
\subsection{Installing Pre-Requisites}\label{installing-pre-requisites}}

For Debian / ZorinOS and likely Ubuntu based systems.

\begin{cautionbox}[TODO]
It would be nice to roll a setup script to take care of this.
\end{cautionbox}

\hypertarget{required}{%
\subsubsection{Required}\label{required}}

\hypertarget{pandoc}{%
\paragraph{Pandoc}\label{pandoc}}

\begin{itemize}
\tightlist
\item
  https://pandoc.org/
\item
  \href{https://github.com/jgm/pandoc/releases/tag/3.8.2.1}{Download}
\end{itemize}

\begin{verbatim}
sudo apt install https://github.com/jgm/pandoc/releases/download/3.8.2.1/pandoc-3.8.2.1-1-amd64.deb
\end{verbatim}

\hypertarget{code-editor}{%
\paragraph{Code Editor}\label{code-editor}}

\begin{establishedbox}[Code Editor]
\href{https://vscodium.com/}{VSCodium} is recommend for privacy
(telemetry/tracking) reasons - https://vscodium.com/
\end{establishedbox}

But any text editor will work. \#\#\#\# make

\begin{verbatim}
sudo apt install make
\end{verbatim}

\hypertarget{jq-and-yq}{%
\paragraph{jq and yq}\label{jq-and-yq}}

\begin{verbatim}
sudo apt install jq yq
\end{verbatim}

\hypertarget{texlive}{%
\paragraph{texlive}\label{texlive}}

\begin{verbatim}
sudo apt install texlive texlive-xetex texlive-latex-extra texlive-fonts-recommended texlive-fonts-extra
\end{verbatim}

\hypertarget{mathjax}{%
\paragraph{MathJax}\label{mathjax}}

\begin{itemize}
\tightlist
\item
  https://www.mathjax.org/
\end{itemize}

\begin{verbatim}
wget https://registry.npmjs.org/mathjax/-/mathjax-3.2.2.tgz
tar xzf mathjax-3.2.2.tgz
mv package/es5/* lib/mathjax
rm -rf package mathjax-3.2.2.tgz
\end{verbatim}

\begin{cautionbox}[TODO]
This need to be rolled into a setup script.
\end{cautionbox}

\hypertarget{optional}{%
\subsubsection{Optional}\label{optional}}

The following are not strictly requires to use this book builder
template.

\hypertarget{obsidian}{%
\paragraph{Obsidian}\label{obsidian}}

\begin{establishedbox}[Highly Recommended]
Editing book chapter content in Obsidian is a very productive means for
editing Markdown source content.
\end{establishedbox}

\begin{itemize}
\tightlist
\item
  https://obsidian.md/
\item
  \href{https://github.com/obsidianmd/obsidian-releases/releases/download/v1.9.14/obsidian_1.9.14_amd64.deb}{Deb
  Package}
\end{itemize}

\begin{verbatim}
sudo apt install https://github.com/obsidianmd/obsidian-releases/releases/download/v1.9.14/obsidian_1.9.14_amd64.deb
\end{verbatim}

\hypertarget{zotero}{%
\paragraph{Zotero}\label{zotero}}

\begin{establishedbox}[Highly Recommended]
If you need to managed citations and references, Zotero integration is
highly recommended.
\end{establishedbox}

\begin{itemize}
\tightlist
\item
  https://www.zotero.org/
\end{itemize}

\begin{verbatim}
sudo cp ./scripts/deps/zotero.list /etc/apt/sources.list.d/
\end{verbatim}

\begin{verbatim}
sudo apt update
\end{verbatim}

\begin{verbatim}
sudo apt install zotero
\end{verbatim}

\hypertarget{better-bibtex-for-zotero}{%
\subparagraph{Better BibTex for Zotero}\label{better-bibtex-for-zotero}}

Install the Better BibTex Plugin for Zotero - Zotero \textgreater{} Tool
\textgreater{} Plugins

\hypertarget{export-citations.bib}{%
\subparagraph{Export citations.bib}\label{export-citations.bib}}

\begin{itemize}
\tightlist
\item
  Zotero \textgreater{} File \textgreater{} Export Library
  \textgreater{} Format: Better BibTeX
\item[$\square$]
  Keep Updated
\item[$\square$]
  Save to: \textasciitilde/Documents/Lib/zotero.bib
\item[$\square$]
  Symlink your \textasciitilde/Documents/Lib/Citations.bib to
  basic-book-builder/lib/zotero.bib
\end{itemize}

\hypertarget{zotero-connector-browser-plugin}{%
\subparagraph{Zotero Connector Browser
Plugin}\label{zotero-connector-browser-plugin}}

\begin{itemize}
\tightlist
\item
  https://chromewebstore.google.com/detail/zotero-connector/ekhagklcjbdpajgpjgmbionohlpdbjgc
\end{itemize}

Provides you the ability to auto add Web resources to your Zotero
citation database.

\hypertarget{lmodern}{%
\paragraph{lmodern}\label{lmodern}}

\begin{verbatim}
sudo apt install lmodern
\end{verbatim}

\hypertarget{epubcheck}{%
\paragraph{epubcheck}\label{epubcheck}}

\begin{verbatim}
sudo apt install epubcheck
\end{verbatim}

\hypertarget{foliate}{%
\paragraph{foliate}\label{foliate}}

An EPub Reader

\begin{itemize}
\tightlist
\item
  https://johnfactotum.github.io/foliate/
\end{itemize}

\begin{verbatim}
sudo apt install https://github.com/johnfactotum/foliate/releases/download/2.6.4/com.github.johnfactotum.foliate_2.6.4_all.deb
\end{verbatim}

\hypertarget{calibre}{%
\paragraph{calibre}\label{calibre}}

An EPub Reader

\begin{itemize}
\tightlist
\item
  https://calibre-ebook.com
\end{itemize}

\begin{verbatim}
sudo apt install calibre
\end{verbatim}

\hypertarget{editing-the-configuration}{%
\subsection{Editing the Configuration}\label{editing-the-configuration}}

\hypertarget{editing-the-book}{%
\section{Editing the Book}\label{editing-the-book}}

\hypertarget{configuration}{%
\subsubsection{Configuration}\label{configuration}}

There are a number of other config files for each format:

\begin{verbatim}
conf/
├── epub-metadata.xml
├── epub_template.html
├── epub.yaml
├── frontmatter_epub.md
├── frontmatter_epub.xhtml
├── frontmatter.html
├── frontmatter.tex
├── header.tex
├── html.yaml
├── latex.yaml
├── markdown.yaml
├── metadata.yaml
├── pandoc.yaml
├── pdf.yaml
├── style.css
└── style_epub.css
\end{verbatim}

\hypertarget{main-config-files}{%
\paragraph{Main Config Files}\label{main-config-files}}

\begin{itemize}
\tightlist
\item
  metadata.yaml - Set Title, etc
\item
  pandoc.yaml - Main Pandoc Config \#\#\#\# Per format Configs
\item
  \texttt{pdf.yaml}
\item
  \texttt{html.yaml}
\item
  \texttt{latex.yaml}
\item
  \texttt{epub.yaml}
\end{itemize}

\hypertarget{frontmatter-config}{%
\subsubsection{FrontMatter Config}\label{frontmatter-config}}

There are 2 Version of the FrontMatter for PDF, and HTML bases formats
that set the Title, Author, Verizon, Copyright, etc\ldots{}

\begin{itemize}
\tightlist
\item
  \texttt{frontmatter.tex}
\item
  \texttt{frontmatter.html}
\item
  \texttt{frontmatter\_epub.*} - Work in Progress
\end{itemize}

\begin{quote}
There is probably a better way to do this.
\end{quote}

\hypertarget{editing-the-content}{%
\subsection{Editing the Content}\label{editing-the-content}}

To edit the book open the \texttt{basic-book-builder} directory as an
Obsidian Vault.

\begin{itemize}
\tightlist
\item
  Edit the Markdown content in the \texttt{chapters} directory.
\end{itemize}

\hypertarget{citations}{%
\subsubsection{Citations}\label{citations}}

\begin{quote}
Note: the Zotero database needs configured to export automatically to
\texttt{lib/citations.bib}
\end{quote}

To insert a Zotero Citation - Ensure the Zotero App and DB are running
on you system. - Alt + I (to insert citation) - Search for and select
citation reference

\hypertarget{usage-building-the-book}{%
\subsection{Usage: Building the Book}\label{usage-building-the-book}}

\hypertarget{pdf}{%
\paragraph{PDF}\label{pdf}}

\begin{verbatim}
make pdf
\end{verbatim}

\hypertarget{html}{%
\paragraph{HTML}\label{html}}

\begin{verbatim}
make html
\end{verbatim}

\hypertarget{latex}{%
\paragraph{LaTex}\label{latex}}

\begin{verbatim}
make latex
\end{verbatim}

\hypertarget{markdown}{%
\paragraph{Markdown}\label{markdown}}

\begin{verbatim}
make markdown
\end{verbatim}

\hypertarget{epub}{%
\paragraph{EPub}\label{epub}}

\begin{quote}
Note: This ePub configuration still needs tuning.
\end{quote}

\begin{verbatim}
make epub
\end{verbatim}

\hypertarget{example-content}{%
\section{Example Content}\label{example-content}}

In \(R\nu\) the
\href{https://en.wikipedia.org/wiki/Planck_units\#Planck_length}{Planck
Length} is the universal unit for measurement of distance, and is
defined approximately to be:
\[\boxed{L_P=\sqrt{\hbar}=5.72928\times10^{-35}m=1 L}\] Where \(1\ L\),
is 1 Planck Length of distance.

\hypertarget{si-conversion-factors}{%
\subsubsection{SI Conversion Factors}\label{si-conversion-factors}}

The following conversion factors can be used to convert observable
quantities of measure from the \emph{SI} system of units to \(R\nu\) to
\textasciitilde6 significant digits.

\begin{longtable}[]{@{}
  >{\raggedright\arraybackslash}p{(\columnwidth - 4\tabcolsep) * \real{0.3256}}
  >{\raggedright\arraybackslash}p{(\columnwidth - 4\tabcolsep) * \real{0.0930}}
  >{\raggedright\arraybackslash}p{(\columnwidth - 4\tabcolsep) * \real{0.5814}}@{}}
\toprule
\begin{minipage}[b]{\linewidth}\raggedright
Conversion Factor
\end{minipage} & \begin{minipage}[b]{\linewidth}\raggedright
Symbol
\end{minipage} & \begin{minipage}[b]{\linewidth}\raggedright
Value
\end{minipage} \\
\midrule
\endhead
meters to Planck Length & \(\chi_P\) &
\(1.74542\times10^{34} \frac{L}{m}\) \\
seconds to Planck Length & \(\tau_p\) &
\(5.23264\times10^{42} \frac{L}{s}\) \\
mass to Planck Length & \(G_P\) & \(1.62871\times10^8 \frac{L}{kg}\) \\
energy to Planck Length & \(E_P\) & \(1.81219\times10^9 \frac{L}{J}\) \\
momentum to Planck Length & \(P_P\) &
\(5.43280\times10^{-1} \frac{L\cdot s}{kg \cdot m}\) \\
temperature to Planck Length & \(k_P\) &
\(2.501998\times10^{-14} \frac{L}{K}\) \\
charge to Planck Length & \(C_P\) &
\(1.89007\times10^{18} \frac{L}{C}\) \\
\bottomrule
\end{longtable}

\hypertarget{physical-constants}{%
\subsubsection{Physical Constants}\label{physical-constants}}

Applying conversion factors from the table above, we can convert SI
values to Reduced Natural Units. For example, performing this analysis
on the the speed of light yields a unit-less number with a value of 1:

\(c = 299792458 \frac{m}{s} = 299792458 \frac{m}{s} \cdot 1.74542\times10^{34} \frac{L}{m} \cdot \frac{1}{5.23264\times10^{42} \frac{L}{s}} = 1.00000\)

\begin{longtable}[]{@{}
  >{\raggedright\arraybackslash}p{(\columnwidth - 6\tabcolsep) * \real{0.2273}}
  >{\raggedright\arraybackslash}p{(\columnwidth - 6\tabcolsep) * \real{0.0909}}
  >{\raggedright\arraybackslash}p{(\columnwidth - 6\tabcolsep) * \real{0.4545}}
  >{\raggedright\arraybackslash}p{(\columnwidth - 6\tabcolsep) * \real{0.2273}}@{}}
\toprule
\begin{minipage}[b]{\linewidth}\raggedright
Quantity
\end{minipage} & \begin{minipage}[b]{\linewidth}\raggedright
Symbol
\end{minipage} & \begin{minipage}[b]{\linewidth}\raggedright
SI
\end{minipage} & \begin{minipage}[b]{\linewidth}\raggedright
\(\nu\)
\end{minipage} \\
\midrule
\endhead
Speed of Light & \(c\) & \(299792458 \frac{m}{s}\) & 1 \\
Reduced Gravitational Constant & \(G_0\) &
\(8.38659\times10^{-10} \frac{m^3}{kg \cdot s^2}\) & 1 \\
Boltzmann's Constant & \(k\) & \(k=1.380649\times10^-23 \frac{J}{K}\) &
1 \\
Permittivity of Free Space & \(\epsilon_o\) &
\(8.854187817620\times10^{-12} \frac{C^{2}s^2}{kg \cdot m^3}\) & 1 \\
Permeability of Free Space & \(\mu_o\) &
\(\huge{\frac{1}{\epsilon_{o} \cdot c^{2}}}\) & 1 \\
Reduced Planck's Constant & \(\hbar\) &
\(1.054571726\times10^-34 \frac{kg \cdot m^2}{s}\) & \(1 L^2\) \\
Mass of the Electron & \(m_e\) & \(9.10938\times10^{-31} kg\) &
\(1.48366\times10^{-22} L\) \\
Charge of the Electron & \(e^-\) & \(-1.60218\times10^{-19} C\) &
\(-3.02822\times10^{-1} L\) \\
Unit Cycle & \(\Theta\) & \(2\pi = 6.28318...\ Radians\) &
\(1 \tau = 6.28318...\ Radians\) \\
\bottomrule
\end{longtable}

\hypertarget{fine-structure-constant}{%
\subsection{Fine Structure Constant}\label{fine-structure-constant}}

As a consistency check, we compute the
\emph{\href{https://en.wikipedia.org/wiki/Fine-structure_constant}{Fine
Structure Constant}} using Reduced Natural Units which is a unit less
ratio that should be independent of our system of units.

\(\huge{\alpha=\frac{e^2}{4\pi\epsilon_o\hbar c}=\frac{e^2}{2\tau}=0.00729735≈\frac{1}{137}}\)

\hypertarget{dimensional-analysis}{%
\paragraph{Dimensional Analysis}\label{dimensional-analysis}}

The reader should be familiar with high school physics and chemistry
\href{https://en.wikipedia.org/wiki/Dimensional_analysis}{dimensional
analysis}.

\begin{itemize}
\tightlist
\item
  \(1\ meter\ (m) = 100\ centimeters\ (cm)\)
\item
  \(1\ kilometer\ (km) = 1000\ meters\ (m)\)
\item
  \(1\ mile = 5280\ feet\ (ft\ or\ ')\)
\item
  \(1\ foot\ (ft\ or\ ') = 12\ inches\ (in\ or\ ")\)
\item
  \(1\ inch\ (") = 2.54\ centimeters\ (cm)\)
\end{itemize}

How many kilometers are in 1 mile?
\(1\ mile = 1\ mile \times \frac{5280\ ft}{mile} \times \frac{12\ in}{ft} \times \frac{2.54\ cm}{in}\times \frac{1\ m}{100 cm} \times \frac{1\ km}{1000 m} = \frac{5280 \times 12 \times 2.54}{100 \times 1000}\ km = \frac{160934.40}{100000}\ km = 1.6\ km\)
Note that each unit in the denominator cancels with one if the numerator
until we are left with only km.

\hypertarget{newtons-law-of-gravity}{%
\subsection{Newton's Law of Gravity}\label{newtons-law-of-gravity}}

The force of gravity (\(F_g\)) between 2 masses, \(m1\) and \(m2\)
separated by distance \(r\) is given by
\href{https://en.wikipedia.org/wiki/Newton\%27s_law_of_universal_gravitation}{Newton's
Law of Gravity}:

\(F_{g} = G \frac{m_{1} m_{2}}{r^{2}}\)

Where \(G\), is the
\href{http://en.wikipedia.org/wiki/Gravitational_Constant}{Gravitational
Constant}.

\(G = 6.67430 \times 10^{-11}\ N\frac{m^2}{kg^2}\)

The strength of gravitational force follow the inverse square law
distributing gravitational flux over the surface area of a sphere
(\(4\pi r^2\)).

\hypertarget{inverse-square-law}{%
\paragraph{Inverse Square Law}\label{inverse-square-law}}

Any source of a signal strength (\(S_0\)) that radiates isotropically in
3-dimensional space will distribute that signal strength (\(S_0\)) over
the surface area of a sphere (\(SA = 4 \pi r\)) of radius (\(r\)). Such
that the intensity (\(I\)) at distance (\(r\)) is:

\[I(r) = \frac{S_0}{4 \pi r^{2}}=\frac{S_0}{2 \tau r^{2}}\]
\includegraphics{lib/img/Inverse_square_law.svg.png} \#\#\#\# \(R\nu\)
Reduced Gravitational Constant In this version of Newton's Law of
Gravity we introduce a new constant \(G_0\), the reduced gravitational
constant to accommodate for the factor of \(4\pi = 2\tau\) which is has
been integrated in the SI version of the gravitational constant.

\(F_g =G \frac{m_{1} m_{2}}{r^{2}}= G_0 \frac{m_{1} m_{2}}{4 \pi r^{2}}=G_0 \frac{m_{1} m_{2}}{2 \tau r^{2}}\)

Where:

\(G = \frac{G_{0}}{2\tau} = 6.67384 \times 10^{-11} \frac{N \cdot m^2}{kg^2}\)

Analyzing the units:
\[\frac{N \cdot m^2}{kg^2} = \left( \frac{\left( kg \cdot \frac{m}{s^2} \right) \cdot m^2}{kg^2} \right)=\frac{m^3}{s^2 kg}\]
Converting seconds to meters with the SI speed of light as a conversion
factor:
\[\frac{m^3}{s^2 kg}\cdot\frac{1}{c^2}=\frac{m^3}{s^2 kg}\cdot\frac{s^2}{m^2}=\frac{m}{kg}\]

Thus where space and time are measured in units of meters, the reduced
gravitational constant, is:

\[\boxed{G_0=\frac{2\tau G}{c^2}=\frac{2\tau \cdot 6.67384 \times 10^{-11}}{299792458^2} \frac{m}{kg} = 9.33135 \times 10^-27 \frac{m}{kg}}\]

\begin{quote}
Observation This implies that not only can space an time be measure in
units of meters, but so can mass.
\end{quote}

\hypertarget{relativistic-energy-momentum-relation}{%
\subsubsection{Relativistic Energy Momentum
Relation}\label{relativistic-energy-momentum-relation}}

Einsteins
\href{https://en.wikipedia.org/wiki/Energy\%E2\%80\%93momentum_relation}{Relativistic
Energy Momentum} relationship shows a Pythagorean relation between the
total energy (\(E\)), rest mass (\(m_0\)) and momentum (\(p\)) of a
system.

\(E^{2} = (m_0 \cdot c^2)^2 + (p \cdot c)^2\)

Where space and time are both measure in units of meters, c=1.

\(E^2=(m_0)^2+(p)^2\)

From this we can see that Energy, Momentum and Mass have equivalent
units.

\begin{quote}
\emph{While we do not really know what energy, mass and momentum are we
know that they are fundamentally ``made'' out of the same stuff because
they have the same units.}
\end{quote}

\hypertarget{objects-of-mass-at-rest}{%
\subparagraph{Objects of mass at rest}\label{objects-of-mass-at-rest}}

For an object at rest with no momentum (\(p = 0\)) we see Einstein's
famous equations:

\(E = m_{0} \cdot c^2\)

Or, with \(c=1\), this is much simpler to understand. Energy = Mass

\(E = m_0\)

\hypertarget{zero-mass-objects-moving-at-the-speed-of-light}{%
\subparagraph{Zero mass objects moving at the speed of
light}\label{zero-mass-objects-moving-at-the-speed-of-light}}

And for objects with no mass, like photos, (\(m_{0}= 0\)):

\(E=pc\)

Or, with \(c=1\), this is much simpler to understand. Energy = Momentum

\(E=p\)

\hypertarget{plancks-constant}{%
\subsection{Planck's Constant}\label{plancks-constant}}

The \href{https://en.wikipedia.org/wiki/Planck_constant}{Reduced Planck
constant} , ħ, represents a conversion factor for relating the
frequency, \(\omega\) (in \(2\pi\) radians per second), of a photon to
the energy of that photon. This can easily be seen from the simple but
profound relationship:

\(E=\hbar\omega\)

Where:

\(\hbar=1.054571726 \times 10^{−34} J \cdot s\)

and

\(J \cdot s = {kg}\cdot\frac{m^2}{s}\)

\begin{quote}
Reduced Planck's Constant \(\hbar = \frac{h}{2\pi} = \frac{h}{\tau}\)
\end{quote}

Simplifying our units by converting time and mass to units of meters:
\[\boxed{\hbar=1.054571726 \times 10^{−34} {kg}\cdot\frac{m^2}{s}\cdot\frac{G_0}{c}=3.282462\times10^{-69}m^2}\]

Which suggest that the Plank constant can be interpreted as an areas for
which the square root of is suspiciously close to the Plank length:

\[\boxed{\sqrt{\hbar}=\sqrt{3.282462\times10^{-69}m^2}=5.72928\times10^{-35}m}\]

\hypertarget{planck-area}{%
\paragraph{Planck Area}\label{planck-area}}

The
\href{https://en.wikipedia.org/wiki/Planck_units\#Derived_units}{Planck
Area} is the square of the
\href{https://en.wikipedia.org/wiki/Planck_units\#Planck_length}{Planck
Length}.

\(l_{P}= \sqrt{\frac{\hbar G}{c^3}}\)

and \(l_{P}^{2}= \frac{\hbar G}{c^3}\)

In \(R\nu\) units both \(c\) and \(G_o\) are 1.

\(l_{P} = \sqrt{\hbar}\)

and \(l_P^{2}=\hbar\) \#\# Bekenstein's Bound After having recently read
\emph{Three Roads to Quantum Gravity} by Lee Smolin, I now suspect the
meaning of this areas is related to the
\href{https://en.wikipedia.org/wiki/Bekenstein_bound}{Bekensteins Law}
as applied to a surface areas surrounding a mass. Where the
\href{https://en.wikipedia.org/wiki/Entropy_(statistical_thermodynamics)}{thermodynamic
entropy}, \emph{S}, is proportional to the the enclosed surface area,
\(A\).

\(S=\frac{1}{4}\cdot\frac{A}{G\hbar}\)

\(S=\frac{k c^{3} A}{4 G \hbar}\)

\(S \le \frac{2\pi k R E}{\hbar c} = \frac{\tau R k E}{\hbar c}\)

From our new values for \(G_0\)and \(\hbar\) we can likely rewrite this:

\(S=\frac{\pi\cdot A}{\hbar G_0}\)

With the limiting case being at the Plank scale.

\(S=\frac{\pi\cdot \sqrt{\hbar}}{\hbar G_0}\)

\hypertarget{planck-length}{%
\subsection{Planck Length}\label{planck-length}}

https://en.wikipedia.org/wiki/Planck\_length

The concept of the Planck Length comes from exploring the limits of
Quantum Mechanics and General Relativity. The limits of General
Relativity can be seen a the event horizon of a black hole, described by
the Schwarzschild Radius. And the limits of Quantum Mechanics can be
found in the Compton Wavelength for a given quanta.

The
\href{https://simple.wikipedia.org/wiki/Schwarzschild_radius}{Schwarzschild
Radius} is defined as the distance at which light cannot escape from the
gravitational field of a mass (m):

Classic Derivation.

\(r_S=\frac{2G m}{c^2}\)

The reduced
\href{https://en.wikipedia.org/wiki/Compton_wavelength}{Compton
Wavelength} represents a lower limit on the wavelength for quanta that
can interact with a quantum particle with mass (m):

\(\lambda_C=\frac{h}{m c}\)

\(\bar{\lambda_C}=\frac{2\pi\hbar}{m c}=\frac{\tau\hbar}{m c}\)

And set the Schwarzschild Radius equal to the Compton Wavelength:
\(r_S=\lambda_C\)

\(\frac{2Gm}{c^{2}}=\frac{h}{m c}\)

\(m^{2}= \frac{hc}{2G}\)

\(m = \sqrt{\frac{hc}{2G}}\)

\(l_P=\frac{2G\sqrt{\frac{hc}{2G}}}{c^2}\)
\(l_P=\frac{2G\sqrt{\frac{hc}{2G}}}{c^{2}}= \sqrt{\frac{2Gh}{c^2}}\)

With reduced Compton Wavelength \(r_S=\bar{\lambda_C}\)

\(\frac{2Gm}{c^{2}}=\frac{\tau\hbar}{m c}\)

\(m^2=\frac{\tau\ \hbar\ c}{2G}\)

\(m = \sqrt{\frac{\tau\ \hbar\ c}{2G}}\)

\(l_P=\frac{2G\sqrt{\frac{\tau\ \hbar\ c}{2G}}}{c^2}\)

\(l_P=\frac{2G\sqrt{\frac{\tau\ \hbar\ c}{2G}}}{c^{2}}=\sqrt{\frac{4\ \tau\ G\ \hbar}{c^3}}\)

If we reduce the units in these equation to those of mass and time
measured in meters.

\(l_P=\sqrt{4\tau\hbar G_{o}}\)

and

\(\lambda_C=\frac{\hbar}{m}\)

\(m=R_s=\lambda_C=\frac{\hbar}{m}\)

This is known as the Planck Mass, \(M_P\). \(M_P=m=\sqrt{\hbar}\)

Solving the Compton Wavelength for distance we find the classic Plank
Length:

\(\lambda_C=\frac{\hbar}{\sqrt{\hbar}}=\frac{\sqrt{\hbar}}{\sqrt{\hbar}}\cdot\frac{\hbar}{\sqrt{\hbar}}=\sqrt{\hbar}=L_P\)

Which is in precise agreement with the value we found in above. Thus the
Plank Length is:

\(L_P=\sqrt{\hbar}=5.72928\times10^{-35}m\)

When we measure distance, time, and mass in units of distance, c=1, and
the Plank Time, \(T_P\), is equal to Plank Length, \(L_P\), which is
equal to the Plank Mass, \(M_P\):

\[\boxed{L_P=T_P=M_P}\]

\begin{longtable}[]{@{}
  >{\raggedright\arraybackslash}p{(\columnwidth - 4\tabcolsep) * \real{0.3000}}
  >{\raggedright\arraybackslash}p{(\columnwidth - 4\tabcolsep) * \real{0.3000}}
  >{\raggedright\arraybackslash}p{(\columnwidth - 4\tabcolsep) * \real{0.4000}}@{}}
\toprule
\begin{minipage}[b]{\linewidth}\raggedright
Conversion Factor
\end{minipage} & \begin{minipage}[b]{\linewidth}\raggedright
Symbol
\end{minipage} & \begin{minipage}[b]{\linewidth}\raggedright
Value
\end{minipage} \\
\midrule
\endhead
meters to Planck Length & \(\chi_P\) &
\(1.74542\times10^{34} \frac{L}{m}\) \\
seconds to Planck Length & \(\tau_p\) &
\(5.23264\times10^{42} \frac{L}{s}\) \\
mass to Planck Length & \(G_P\) & \(1.62871\times10^8 \frac{L}{kg}\) \\
energy to Planck Length & \(E_P\) & \(1.81219\times10^9 \frac{L}{J}\) \\
momentum to Planck Length & \(P_P\) &
\(5.43280\times10^{-1} \frac{L\cdot s}{kg \cdot m}\) \\
temperature to Planck Length & \(k_P\) &
\(2.501998\times10^{-14} \frac{L}{K}\) \\
charge to Planck Length & \(C_P\) &
\(1.89007\times10^{18} \frac{L}{C}\) \\
\bottomrule
\end{longtable}

Applying conversion factors from the table above, we can convert SI
values to Reduced Natural Units. \(c=\frac{1}{\sqrt{\epsilon_o \mu_o}}\)

\begin{longtable}[]{@{}
  >{\raggedright\arraybackslash}p{(\columnwidth - 6\tabcolsep) * \real{0.2308}}
  >{\raggedright\arraybackslash}p{(\columnwidth - 6\tabcolsep) * \real{0.2308}}
  >{\raggedright\arraybackslash}p{(\columnwidth - 6\tabcolsep) * \real{0.2308}}
  >{\raggedright\arraybackslash}p{(\columnwidth - 6\tabcolsep) * \real{0.3077}}@{}}
\toprule
\begin{minipage}[b]{\linewidth}\raggedright
Quantity
\end{minipage} & \begin{minipage}[b]{\linewidth}\raggedright
Symbol
\end{minipage} & \begin{minipage}[b]{\linewidth}\raggedright
SI
\end{minipage} & \begin{minipage}[b]{\linewidth}\raggedright
\(\nu\)
\end{minipage} \\
\midrule
\endhead
Speed of Light & \(c\) & \(299792458 \frac{m}{s}\) & 1 \\
Gravitational Constant & \(G_0\) &
\(8.38659\times10^{-10} \frac{m^3}{kg \cdot s^2}\) & 1 \\
Boltzmann's Constant & \(k\) & \(k=1.380649\times10^-23 \frac{J}{K}\) &
1 \\
Permittivity of Free Space & \(\epsilon_o\) &
\(8.854187817620\times10^{-12} \frac{C^{2}s^2}{kg \cdot m^3}\) & 1 \\
Permeability of Free Space & \(\mu_o\) &
\(\huge{\frac{1}{\epsilon_{o} \cdot c^{2}}}\) & 1 \\
Planck's Constant & \(\hbar\) &
\(1.054571726\times10^-34 \frac{kg \cdot m^2}{s}\) & \(1 L^2\) \\
Mass of the Electron & \(m_e\) & \(9.10938\times10^{-31} kg\) &
\(1.48366\times10^{-22} L\) \\
Charge of the Electron & \(e^-\) & \(-1.60218\times10^{-19} C\) &
\(-3.02822\times10^{-1} L\) \\
\bottomrule
\end{longtable}

\hypertarget{fine-structure-constant-1}{%
\subsection{Fine Structure Constant}\label{fine-structure-constant-1}}

https://en.wikipedia.org/wiki/Fine-structure\_constant As a consistency
check, we compute the \emph{Fine Structure Constant} using Reduced
Natural Units which is a unit less ratio that should be independent of
our system of units.

\(\huge{\alpha=\frac{e^2}{4\pi\epsilon_o\hbar c}=\frac{e^2}{4\pi}=0.00729735≈\frac{1}{137}}\)

This check confirms that our system of Reduced Natural Units has
internally consistent values for \(c\), \(\epsilon_o\), \(\hbar\) and
\(e-\). And also \(G_o\) which was used to computer prior values is also
consistent.

\hypertarget{sage-code}{%
\subsection{Sage Code}\label{sage-code}}

Unit Analysis computations have been performed with
\href{https://www.sagemath.org/}{Sage Math}.

\begin{Shaded}
\begin{Highlighting}[]
\CommentTok{\# Define constance}
\ExtensionTok{one}\NormalTok{ = 1.n}\ErrorTok{(}\VariableTok{digits}\OperatorTok{=}\NormalTok{6}\KeywordTok{)}
\ExtensionTok{pi}\NormalTok{ = pi.n}\ErrorTok{(}\VariableTok{digits}\OperatorTok{=}\NormalTok{6}\KeywordTok{)}
\ExtensionTok{tau}\NormalTok{ = 2 }\PreprocessorTok{*}\NormalTok{ pi}
\ExtensionTok{t}\NormalTok{ = tau}

\CommentTok{\# Define the units}
\ExtensionTok{meters}\NormalTok{ = var}\ErrorTok{(}\StringTok{\textquotesingle{}m\textquotesingle{}}\KeywordTok{)}
\ExtensionTok{m}\NormalTok{ = one}\PreprocessorTok{*}\NormalTok{meters}

\ExtensionTok{seconds}\NormalTok{ = var}\ErrorTok{(}\StringTok{\textquotesingle{}s\textquotesingle{}}\KeywordTok{)}
\ExtensionTok{s}\NormalTok{ = seconds}

\ExtensionTok{kilograms}\NormalTok{ = var}\ErrorTok{(}\StringTok{\textquotesingle{}kg\textquotesingle{}}\KeywordTok{)}
\ExtensionTok{kg}\NormalTok{ = kilograms}

\ExtensionTok{newtons}\NormalTok{ = kg }\PreprocessorTok{*}\NormalTok{ m / s\^{}2}
\ExtensionTok{N}\NormalTok{ = newtons}

\ExtensionTok{joules}\NormalTok{ = N }\PreprocessorTok{*}\NormalTok{ m}
\ExtensionTok{J}\NormalTok{ = joules}

\ExtensionTok{print}\ErrorTok{(}\StringTok{"pi    ="}\ExtensionTok{,}\NormalTok{ pi}\KeywordTok{)}
\ExtensionTok{print}\ErrorTok{(}\StringTok{"tau   ="}\ExtensionTok{,}\NormalTok{ t}\KeywordTok{)}

\CommentTok{\# Speed of light in meters/second}
\ExtensionTok{speed\_of\_light}\NormalTok{ = 299792458 }\PreprocessorTok{*}\NormalTok{ meters/seconds}
\ExtensionTok{sol}\NormalTok{ = speed\_of\_light}
\ExtensionTok{c}\NormalTok{ = sol}
\ExtensionTok{print}\ErrorTok{(}\StringTok{"si c  ="}\ExtensionTok{,}\NormalTok{ c}\KeywordTok{)}

\ExtensionTok{rnu\_c}\NormalTok{ = c / c}
\ExtensionTok{print}\ErrorTok{(}\StringTok{"R\textbackslash{}u03BD c  ="}\ExtensionTok{,}\NormalTok{ rnu\_c}\KeywordTok{)}

\CommentTok{\# Gravitational Constant}
\ExtensionTok{gravitational\_constant}\NormalTok{ = 6.67384e{-}11 }\PreprocessorTok{*}\NormalTok{ N}\PreprocessorTok{*(}\NormalTok{m\^{}2/kg\^{}2}\PreprocessorTok{)}
\ExtensionTok{G}\NormalTok{ = gravitational\_constant}
\ExtensionTok{print}\ErrorTok{(}\StringTok{"si G  ="}\ExtensionTok{,}\NormalTok{ G}\KeywordTok{)}

\ExtensionTok{rnu\_G}\NormalTok{ = 4}\PreprocessorTok{*}\NormalTok{pi}\PreprocessorTok{*}\NormalTok{G/c\^{}2}
\ExtensionTok{Go}\NormalTok{ = rnu\_G}
\ExtensionTok{print}\ErrorTok{(}\StringTok{"R\textbackslash{}u03BD Go ="}\ExtensionTok{,}\NormalTok{ Go}\KeywordTok{)}


\CommentTok{\# Planck\textquotesingle{}s Constant}
\ExtensionTok{reduced\_plancks\_constant}\NormalTok{ = 1.054571726e{-}34 }\PreprocessorTok{*}\NormalTok{ J}\PreprocessorTok{*}\NormalTok{s}
\ExtensionTok{h\_bar}\NormalTok{ = reduced\_plancks\_constant}
\ExtensionTok{print}\ErrorTok{(}\StringTok{"si \textbackslash{}u210F  ="}\ExtensionTok{,}\NormalTok{ h\_bar}\KeywordTok{)}

\ExtensionTok{rnu\_h\_bar}\NormalTok{ = h\_bar }\PreprocessorTok{*}\NormalTok{ Go / c}
\ExtensionTok{print}\ErrorTok{(}\StringTok{"R\textbackslash{}u03BD "}\ExtensionTok{u}\StringTok{"\textbackslash{}u210F  ="}\ExtensionTok{,}\NormalTok{ rnu\_h\_bar}\KeywordTok{)}

\CommentTok{\# Planck Length}
\ExtensionTok{rnu\_h\_bar\_str}\NormalTok{ = str}\ErrorTok{(}\ExtensionTok{rnu\_h\_bar}\KeywordTok{)} 
\ExtensionTok{numerical\_part\_str}\NormalTok{ = rnu\_h\_bar\_str.split}\ErrorTok{(}\StringTok{\textquotesingle{}*\textquotesingle{}}\KeywordTok{)}\ExtensionTok{[0]}  
\ExtensionTok{numerical\_part\_str}\NormalTok{ = numerical\_part\_str.strip}\ErrorTok{(}\StringTok{\textquotesingle{}()\textquotesingle{}}\KeywordTok{)}
\ExtensionTok{numerical\_part}\NormalTok{ = float}\ErrorTok{(}\ExtensionTok{numerical\_part\_str}\KeywordTok{)}
\ExtensionTok{rnu\_sqrt\_h\_bar}\NormalTok{ = numerical\_part\^{}}\ErrorTok{(}\ExtensionTok{1/2}\KeywordTok{)}
\CommentTok{\# \^{} Sage cannot process sqrt on units... Lame.}
\ExtensionTok{lP}\NormalTok{ = rnu\_sqrt\_h\_bar }\PreprocessorTok{*}\NormalTok{ m}
\ExtensionTok{print}\ErrorTok{(}\StringTok{"R\textbackslash{}u03BD \textbackslash{}u221A\textbackslash{}u210F ="}\ExtensionTok{,}\NormalTok{ lP}\KeywordTok{)}
\end{Highlighting}
\end{Shaded}

\hypertarget{output}{%
\paragraph{Output}\label{output}}

\begin{verbatim}
pi    = 3.14159
tau   = 6.28319
si c  = 299792458*m/s
Rν c  = 1
si G  = (6.67384e-11)*m^3/(kg*s^2)
Rν Go = (9.33135e-27)*m/kg
si ℏ  = (1.05457e-34)*kg*m^2/s
Rν ℏ  = (3.28246e-69)*m^2
Rν √ℏ = (5.72928e-35)*m
si lP = (1.61620e-35)*sqrt(m^2)
Rν lP = (2.77455e-47)*sqrt(m^3/kg)
\end{verbatim}

\hypertarget{terminology}{%
\section{Terminology}\label{terminology}}

\hypertarget{citations-1}{%
\section*{Citations}\label{citations-1}}
\addcontentsline{toc}{section}{Citations}

\hypertarget{refs}{}
\begin{CSLReferences}{1}{0}
\leavevmode\vadjust pre{\hypertarget{ref-JohnHaverlackACEP}{}}%
{``John {Haverlack} \textbar{} {ACEP}.''} n.d.
https://www.uaf.edu/acep/about/our-team/john-haverlack.php. Accessed
September 30, 2025.

\end{CSLReferences}

\end{document}
